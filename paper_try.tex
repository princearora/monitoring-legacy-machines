%% This is file `elsarticle-template-5-harv.tex',
%%
%% Copyright 2009 Elsevier Ltd
%%
%% This file is part of the 'Elsarticle Bundle'.
%% ---------------------------------------------
%%
%% It may be distributed under the conditions of the LaTeX Project Public
%% License, either version 1.2 of this license or (at your option) any
%% later version.  The latest version of this license is in
%%    http://www.latex-project.org/lppl.txt
%% and version 1.2 or later is part of all distributions of LaTeX
%% version 1999/12/01 or later.
%%
%% The list of all files belonging to the 'Elsarticle Bundle' is
%% given in the file `manifest.txt'.
%%
%% Template article for Elsevier's document class `elsarticle'
%% with harvard style bibliographic references
%%
%% $Id: elsarticle-template-5-harv.tex 159 2009-10-08 06:08:33Z rishi $
%% $URL: http://lenova.river-valley.com/svn/elsbst/trunk/elsarticle-template-5-harv.tex $
%%
\documentclass[preprint,authoryear,5p,times,twocolumn]{elsarticle}

%% Use the option review to obtain double line spacing
%% \documentclass[authoryear,preprint,review,12pt]{elsarticle}

%% Use the options 1p,twocolumn; 3p; 3p,twocolumn; 5p; or 5p,twocolumn
%% for a journal layout:
%% \documentclass[final,authoryear,1p,times]{elsarticle}
%% \documentclass[final,authoryear,1p,times,twocolumn]{elsarticle}
%% \documentclass[final,authoryear,3p,times]{elsarticle}
%% \documentclass[final,authoryear,3p,times,twocolumn]{elsarticle}
%% \documentclass[final,authoryear,5p,times]{elsarticle}
%% \documentclass[final,authoryear,5p,times,twocolumn]{elsarticle}

%% if you use PostScript figures in your article
%% use the graphics package for simple commands
%% \usepackage{graphics}
%% or use the graphicx package for more complicated commands
%% \usepackage{graphicx}
%% or use the epsfig package if you prefer to use the old commands
%% \usepackage{epsfig}

%% The amssymb package provides various useful mathematical symbols
\usepackage{amssymb}
%% The amsthm package provides extended theorem environments
%% \usepackage{amsthm}

%% The lineno packages adds line numbers. Start line numbering with
%% \begin{linenumbers}, end it with \end{linenumbers}. Or switch it on
%% for the whole article with \linenumbers after \end{frontmatter}.
%% \usepackage{lineno}

%% natbib.sty is loaded by default. However, natbib options can be
%% provided with \biboptions{...} command. Following options are
%% valid:

%%   round  -  round parentheses are used (default)
%%   square -  square brackets are used   [option]
%%   curly  -  curly braces are used      {option}
%%   angle  -  angle brackets are used    <option>
%%   semicolon  -  multiple citations separated by semi-colon (default)
%%   colon  - same as semicolon, an earlier confusion
%%   comma  -  separated by comma
%%   authoryear - selects author-year citations (default)
%%   numbers-  selects numerical citations
%%   super  -  numerical citations as superscripts
%%   sort   -  sorts multiple citations according to order in ref. list
%%   sort&compress   -  like sort, but also compresses numerical citations
%%   compress - compresses without sorting
%%   longnamesfirst  -  makes first citation full author list
%%
%% \biboptions{longnamesfirst,comma}

% \biboptions{}

\journal{CIRP Annals - Manufacturing Technology}

\begin{document}

\begin{frontmatter}

%% Title, authors and addresses

%% use the tnoteref command within \title for footnotes;
%% use the tnotetext command for the associated footnote;
%% use the fnref command within \author or \address for footnotes;
%% use the fntext command for the associated footnote;
%% use the corref command within \author for corresponding author footnotes;
%% use the cortext command for the associated footnote;
%% use the ead command for the email address,
%% and the form \ead[url] for the home page:
%%
%% \title{Title\tnoteref{label1}}
%% \tnotetext[label1]{}
%% \author{Name\corref{cor1}\fnref{label2}}
%% \ead{email address}
%% \ead[url]{home page}
%% \fntext[label2]{}
%% \cortext[cor1]{}
%% \address{Address\fnref{label3}}
%% \fntext[label3]{}

\title{Intelligent Monitoring of Legacy Machines in Manufacturing Industries}

%% use optional labels to link authors explicitly to addresses:
%% \author[label1,label2]{<author name>}
%% \address[label1]{<address>}
%% \address[label2]{<address>}

\author{Prince Arora}
\address{Indian Institute of Technology Madras, INDIA}

\author{Dr. Athulan Vijayaraghavan}
\address{System Insights Inc.}
\begin{abstract}
%% Text of abstract
Even though MTConnect has grown fast, a big hurdle is that the majority of machines in use today are legacy machines. They do not spew alarms, events or conditions as the relatively new MTConnect compatible machines do. It is equally important to provide meaningful insights for these machines. A quick and dirty way to achieve this can be to mount external sensors and retrieve data from the machine in MTConnect format. This way all machines, even the manually controlled ones can be brought under the aegis of MTConnect.
\end{abstract}

\begin{keyword}
%% keywords here, in the form: keyword \sep keyword
mtconnect \sep legacy machines \sep manufacturing \sep in-situ analysis
%% MSC codes here, in the form: \MSC code \sep code
%% or \MSC[2008] code \sep code (2000 is the default)

\end{keyword}

\end{frontmatter}

% \linenumbers

%% main text
\section{Introduction}
\label{Introduction}
Why did I do it?
What did I find out?
What does it mean?
What next?

Manufacturing Industries are the unacclaimed giants supporting our present world built on the pillars of consumerism. They use up a great amount of our natural and human resources to produce sell-able products. It has  always been important to keep the resources they use up in check. It has become more important today owing to the ever increasing cost of energy, raw materials and man power. Hence, it has become imminent that we make optimal use of all resources avaiable to us. But how do we optimize resources in a manufacturing industry. There are quite a few factors which hamper such attempts:
\begin{itemize}
\item Lack of Data: there exists very little data on the actual run time of machine.
\item Lack of Standards: with thousands of manufacturers, any attempt of data acquisition gets right down into the bin.
\item Lack of Awareness: manufacturers don’t know if there is any other way out apart from the age old method of operations management.
\end{itemize}
Though awareness is bound to creep in sooner or later, the first two listed setbacks are ones which hamper the implementation of any optimization method on a large scale. Standardization would lead to development of better techniques to capture data, which in turn would lead to better avaiablility of data. If standardized, the act of data collection would be simpler and might lead to path breaking innovations in the manufacturing factor. In a past few years, this is what MTConnect\footnote{MTConnect is an upcoming manufacturing industry standard to facilitate the organized retrieval of process information from numerically controlled machine tools. It is a lightweight, open, and extensible protocol designed for the exchange of data between shop floor equipment and software applications. Defined as a read-only standard, it presents data from shop floor devices in XML format.} has been trying to achieve.

The advent and widespread acceptance of such a standard can open many new avenues in the field of manufacturing management. Apart from monitoring the machines, for the first time ever, it might be possible to make advancements in the not so well developed field of predictive quality and predictive maintenance. But for all this to materialize, it is important to bring most of our current machines under the aegis of a single standard. It is easy to write adapters for the new numerically computer controlled machines but getting meaningful data from old legacy machines still seems a task in distant future. We believe it might make a difference if we could provide some architecture to gather from these machines data that makes sense. Hence we tried to come up with a new pipeline to get data and generate information in-situ. The later sections explain it in detail.

\section{Approach}
\label{Approach}
\subsection{Framework}
Past efforts in this field have been targeted at getting data by mounting sensors and then analyzing it on a remote machine. The new architecture proposed by us tries to do the same in-situ at the point of data collection itself. We call this iAdapter or the intelligent adapter as it provides comprehensible information using data spewed by comparatively cheap sensors.

\begin{list}{-}{The iAdapter hardware framework is conceived as follows:}
\item A set of analog sensors mounted over the machine \item A Labjack U3 device to take input from the legacy/low-end analog sensors \item A ConnectOne device to take  input from compatible sensors* \item An embedded system to take input from the Labjack device* \item Intra/Internet to transmit data 
\end{list}
(the requirements marked with * can be grouped together in a single device)

\subsection{Sensors}
Various different types of sensors can be mounted in conjunction with the labjack device. The most important ones amongst those used for our experiments included 3-axis accelerometers at the spindle head and at the body and  hydraulic and pneumatic sensors to monitor the system and coolant pressure. The accelerometers were connected to the Labjack device whereas the rest of them were connected via ConnectOne\footnote{ConnectOne is a proprietary data acquisition device developed by System Insights Inc., a company working with MTConnect Compatible Machines to give insights about the manufacturing processes.}.

\subsection{The Black Box}
The analysis blackbox in this case can be any full fledged PC or even an embedded system. In order to test it out with the bare bones, it was tried to implement all the developement codes on an embedded system named Alekto, an ARM9 RISC embedded industrial computer. It accepts raw sensor data from a variety of devices as well as the regular MTConnect agent stream, and tries to make sense out of it.

In general the amount of data generated is huge, and it is important to scale it down before sending it over to the main database or the MTConnect Agent. It performs onsite down sampling and sends over the relevant information to the remote computer/cloud storage. This helps improve performance characteristics, limits bandwidth usage and helps maintan a fixed latency between data packets. The second big advantage it provides over the current architecture is the ability to define alarm conditions. Given a set of limits, it can keep on spewing information regarding device and parameter conditions. With assissted learning, it can also improve upon these limits over time.

\section{Implementation}
\label{Implementation}
\subsection{Data Acquisition}
The sensors were mounted over a TAKISAWA TC-200 horizontal Lathe. A CNC lathe machine was chosen so that the results obtained are applicable to a wide range of machines across various industries. Such machines are used for a variety of turning operations all across the world. The sensors chosen for the task were also pretty generic. They included a hydraulic and pneumatic pressure sensors and 3-axis accelerometers. The later were picked up from a commonly accessible hobby robotics online shop.

The analog sensors were connected to a Labjack U3-HV\footnote{Labjack U3-HV is a USB based multifunction data acquisition and control device which allows for upto 16 analog input channels.}. It can be interfaced with any analog or digital sensor to monitor condition of the machine in realtime. The data collected is transferred over USB and any embedded system can be used to log the same. Here it is worth mentioning that it is required to separately compile the adapter for Labjack device individually for each type of embedded device.

Even after the driver is ready, we need a software MTConnect adapter to transmit data following the MTConnect protocols. The Labjack adapter was coded in C++. It follows the standard framework of other MTConnect Adapters currently in use.  The adapter was initially programmed to record and transmit one value per port. Later it was decided to use the same to take sensor input at around 1000Hz per channel. Hence, the adapter was written from scratch by making use of the stream mode supported by the device. As of now, the adapter has been tested for the 1-7 channel input at around 1000Hz per channel.

The adapter is installed along with the driver on Alekto, an ARM9 RISC embedded industrial computer. The Labjack U3-HV connects via USB to an external USB hub powered by external power.

\subsection{Data Analysis}
The Labjack adapter explained in the last subsection spews analog values of the sensor data at a specified network socket. The data acquired through various sources is then analyzed on Alekto. The amount of data generated is huge, and it is important to scale it down before sending it over to the main database or the MTConnect Agent. This is but one function performed at this stage. The function which sets it apart is its ability to find alarm states with the sensor data. This feature helps bring in Legacy Machines under the aegis of MTConnect Compatibility and can potentially help in predictive/preventive maintenance.

Two different languages Ruby and R were used in this particular implementation of iAdapter.
%% The Appendices part is started with the command \appendix;
%% appendix sections are then done as normal sections
%% \appendix
%% \section{}
%% \label{}

%% References
%%
%% Following citation commands can be used in the body text:
%%
%%  \citet{key}  ==>>  Jones et al. (1990)
%%  \citep{key}  ==>>  (Jones et al., 1990)
%%
%% Multiple citations as normal:
%% \citep{key1,key2}         ==>> (Jones et al., 1990; Smith, 1989)
%%                            or  (Jones et al., 1990, 1991)
%%                            or  (Jones et al., 1990a,b)
%% \cite{key} is the equivalent of \citet{key} in author-year mode
%%
%% Full author lists may be forced with \citet* or \citep*, e.g.
%%   \citep*{key}            ==>> (Jones, Baker, and Williams, 1990)
%%
%% Optional notes as:
%%   \citep[chap. 2]{key}    ==>> (Jones et al., 1990, chap. 2)
%%   \citep[e.g.,][]{key}    ==>> (e.g., Jones et al., 1990)
%%   \citep[see][pg. 34]{key}==>> (see Jones et al., 1990, pg. 34)
%%  (Note: in standard LaTeX, only one note is allowed, after the ref.
%%   Here, one note is like the standard, two make pre- and post-notes.)
%%
%%   \citealt{key}          ==>> Jones et al. 1990
%%   \citealt*{key}         ==>> Jones, Baker, and Williams 1990
%%   \citealp{key}          ==>> Jones et al., 1990
%%   \citealp*{key}         ==>> Jones, Baker, and Williams, 1990
%%
%% Additional citation possibilities
%%   \citeauthor{key}       ==>> Jones et al.
%%   \citeauthor*{key}      ==>> Jones, Baker, and Williams
%%   \citeyear{key}         ==>> 1990
%%   \citeyearpar{key}      ==>> (1990)
%%   \citetext{priv. comm.} ==>> (priv. comm.)
%%   \citenum{key}          ==>> 11 [non-superscripted]
%% Note: full author lists depends on whether the bib style supports them;
%%       if not, the abbreviated list is printed even when full requested.
%%
%% For names like della Robbia at the start of a sentence, use
%%   \Citet{dRob98}         ==>> Della Robbia (1998)
%%   \Citep{dRob98}         ==>> (Della Robbia, 1998)
%%   \Citeauthor{dRob98}    ==>> Della Robbia


%% References with bibTeX database:

\bibliographystyle{model5-names}
\bibliography{<your-bib-database>}

%% Authors are advised to submit their bibtex database files. They are
%% requested to list a bibtex style file in the manuscript if they do
%% not want to use model5-names.bst.

%% References without bibTeX database:

% \begin{thebibliography}{00}

%% \bibitem must have one of the following forms:
%%   \bibitem[Jones et al.(1990)]{key}...
%%   \bibitem[Jones et al.(1990)Jones, Baker, and Williams]{key}...
%%   \bibitem[Jones et al., 1990]{key}...
%%   \bibitem[\protect\citeauthoryear{Jones, Baker, and Williams}{Jones
%%       et al.}{1990}]{key}...
%%   \bibitem[\protect\citeauthoryear{Jones et al.}{1990}]{key}...
%%   \bibitem[\protect\astroncite{Jones et al.}{1990}]{key}...
%%   \bibitem[\protect\citename{Jones et al., }1990]{key}...
%%   \harvarditem[Jones et al.]{Jones, Baker, and Williams}{1990}{key}...
%%

% \bibitem[ ()]{}

% \end{thebibliography}

\end{document}

%%
%% End of file `elsarticle-template-5-harv.tex'.
